\documentclass[a4paper]{article}

\usepackage[utf8]{inputenc}
\usepackage[T1]{fontenc}
\usepackage{textcomp}
\usepackage{amsmath, amssymb}
\usepackage{enumitem}


% figure support
\usepackage{import}
\usepackage{xifthen}
\pdfminorversion=7
\usepackage{pdfpages}
\usepackage{transparent}
\newcommand{\incfig}[1]{%
	\def\svgwidth{\columnwidth}
	\import{./figures/}{#1.pdf_tex}
}

\pdfsuppresswarningpagegroup=1

\title{Introduction to Complexity and Computability - Homework 5}
\date{\today}
\author{Andrew McIsaac}

\begin{document}
\maketitle

\subsection*{Homework Problem 1}
Show that the problem COVER-ORIENTED-CYCLES is NP-complete.
\newline

First show that it is in NP. Create a verifier $V$, which, on input
$\langle \langle G, k \rangle, S \rangle$, where $S$ is the certificate for the
problem, for every vertex in $S$, removes edges incident from the vertex and the
vertex from the graph G.

On the new graph, use depth-first search to traverse the graph and check for an
oriented cycle. This can be done in polynomial time using recursion and keeping
a stack of visited vertices for every vertex, so that if any vertex is visited
more than once in a given vertex's recursion stack (hence there is a cycle) the
algorithm can return that there exists a cycle. Therefore $V$ rejects, as there
is a cycle which has all of its vertices not in $S$. If there is no cycle, all
possible oriented cycles must be covered by $S$, and the verifier accepts (it
also checks that $|S| \leq k$).
\newline

Now use a reduction from VERTEX-COVER to show that COVER-ORIENTED-CYCLES is
NP-hard.

Given a graph $G=(V,E)$, $k$, and $S$ from VERTEX-COVER, we want a $G'=(V',E')$,
$k'$, and $S'$, such that $S$ contains at least one vertex from every edge in
$G$ if and only if $S'$ contains at least one vertex from every oriented cycle
in $G'$. Convert $G$ into an oriented graph by taking every edge $\{u, v\} \in
E$ and creating directed edges in both directions, so that $\{u \mapsto v\} \in
E'$ and $\{v \mapsto u\} \in E'$.

Define $V'=V$, $k'=k$, and $S'=S$. Then if $S$ is a yes-instance of VERTEX-COVER
then no cycle will be in $G'$ that does not have a vertex from $S'$ because
every edge in $G$ has a vertex in $S$ and we do not introduce any new edges
between different vertices that were not in $E$ in some direction.

If $S$ is a no-instance of VERTEX-COVER then in $G$ there will be (at least) a
2-cycle between two vertices that share an incident edge but are not in $S$, and
in $G'$ the same two vertices will not be in $S'$. Since every edge is converted
to a 2-cycle by the reduction there is an oriented cycle that contains no
vertices from $S'$ in $G'$, and $S'$ is a no-instance of COVER-ORIENTED-CYCLES.
\newline

Finally, it is clear that the reduction can be carried out in polynomial time,
as the conversion of the graph can be done in time linear with respect to $E$.

\subsection*{Homework Problem 2}

\begin{enumerate}[label=(\alph*)]
	\item Show that the PARTITION-PROBLEM is polynomially reducible to the
		KNAPSACK problem.

		Let the KNAPSACK problem have sizes $s'(a) = v'(a) = s(a)$ where $s(a)$
		are the values associated with each $a \in A$ from PARTITION-PROBLEM.
		Let capacity $C$ and value $V$ be $C = V = \frac{1}{2}\sum_{a\in A} s(a)$,
		and let the subset $A'$ from KNAPSACK be identical to the subset $A'$
		from PARTITION-PROBLEM. The sets $A$ are identical in both problems
		as well.

		Then for the PARTITION-PROBLEM there is $A' \subseteq A$ such that \\
		$\sum_{a \in A'} s(a) = \sum_{a \in A \setminus A'} s(a)$ if and only if
		for the KNAPSACK problem there is $A' \subseteq A$ such that
		$\sum_{a \in A'} s'(a) \leq C$ and $\sum_{a \in A'} v'(a) \geq V$.

		For a yes-instance, $A'$ has capacity exactly $C$ and value exactly $V$.
		That is, $\sum_{a \in A'} s'(a) = C$ and $\sum_{a \in A'} v'(a) = V$.
		
		For a no-instance, either the capacity of $A'$ is larger than $C$ or
		the value of $A'$ is smaller than $V$, breaking one of the constraints.
		
		It is clear that this reduction can be done in polynomial time simply
		by assigning values to $C$ and $V$ and keeping other variables the same.

	\item Show that the PARTITION-PROBLEM is polynomially reducible to the
		SCHEDULING problem.

		Let the number of processors $m = 2$. Let the set of tasks $T$ be equal
		to the set $A$ from PARTITION-PROBLEM, and $d(x) = s(a)$ for every
		respective element of the respective sets $T$ and $A$.
		Let $D = \frac{1}{2} \sum_{x \in T} d(x)$ (that is, half the sum of the
		duration of all tasks $x$ in $T$). 

		Then the reduction from PARTITION-PROBLEM assigns $A'$ to $T_1$ and
		$A \setminus A'$ to $T_2$. If $A'$ is a yes-instance for
		PARTITION-PROBLEM then for both $T_1$ and $T_2$ $\sum_{x \in T_i} d(x)
		= D$. If $A'$ is a no-instance then one of $T_1$ or $T_2$ will have
		a sum of durations larger than $D$, thus breaking the constraint for the
		SCHEDULING problem. So $A' \subseteq A$ such that $\sum_{a \in A'} s(a)
		= \sum_{a \in A \setminus A'} s(a)$ is in PARTITION-PROBLEM if and only
		if $T$ is partitioned into $m$ parts $T_1, \ldots, T_m$ such that
		$\sum_{x \in T_i} d(x) \leq D$ for every $1 \leq i \leq m$ in SCHEDULING.

		Finally, it is obvious that the reduction is done in polynomial time.

\end{enumerate}

\end{document}
