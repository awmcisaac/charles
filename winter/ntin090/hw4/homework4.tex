\documentclass[a4paper]{article}

\usepackage[utf8]{inputenc}
\usepackage[T1]{fontenc}
\usepackage{textcomp}
\usepackage{amsmath, amssymb}
%\usepackage{amsmath}
\usepackage{enumitem}


% figure support
\usepackage{import}
\usepackage{xifthen}
\pdfminorversion=7
\usepackage{pdfpages}
\usepackage{transparent}
\newcommand{\incfig}[1]{%
	\def\svgwidth{\columnwidth}
	\import{./figures/}{#1.pdf_tex}
}

\pdfsuppresswarningpagegroup=1

\title{Introduction to Computability and Complexity - Homework 4}
\date{\today}
\author{Andrew McIsaac}

\begin{document}
\maketitle

Determine the correct inclusion relationship between the following pairs of
classes. That is, fill in one of the relation symbols $\subsetneq, \subseteq, =,
\supseteq, \supsetneq$, or ? (if we cannot say based on our knowledge). Justify
your answers.
\begin{enumerate}[label=(\alph*)] 

	%%  (a)  %%
	\item SPACE($n$) ? TIME($2^{\log^{3}n}$)

		If we want to show any subset relationship from SPACE to TIME, we would
		have to go through Theorem $1.\text{iii}$ and the Corollary to Theorem
		2. But NSPACE($n$) $\not\subseteq$ TIME($2^{\log^{3}n}$), as $n \notin
		o(\log^{3}n)$.

		If we want to show any subset relationship from TIME to SPACE, we would
		have to at least go through SPACE($2^{\log^{3}n}$) (either by 1.i or by
		1.ii and 1.iv), and then by Theorem 4 show that
		$2^{\log^{3}n} \in o(n)$, which is false.

		Therefore we cannot say the inclusion relationship.

	%%  (b)  %%
	\item TIME($2^{\log^{3}n}$) $\supsetneq$ NSPACE($\log^{2}n$)
		$$
		\text{NSPACE}(\log^{2}n) \overset{(1)}{\subseteq} 
		\text{TIME}(2^{\log^{2.5}n})
		\overset{(2)}{\subsetneq} \text{TIME}(2^{\log^{3}n})
		$$
		(1) By Corollary. $\log^{2}n \in o(\log^{2.5}n)$ \\
		(2) By Theorem 5. 
		$$
		2^{\log^{2.5}n} \in o\left(\frac{2^{\log^3n}}{\log^3n}\right)
		\iff 
		\lim_{n \to \infty} \frac{2^{\log^{2.5}n} \cdot \log^3n}{2^{\log^3n}}=0
		$$
		$2^{\log^3n}$ dominates, so at the limit the equation holds.

	%%  (c)  %%
	\item NSPACE($\log^{2}n$) $\subsetneq$ NTIME($2^{\log^{3}n}$)
		$$
		\text{NSPACE}(\log^2n)\overset{(1)}{\subsetneq} \text{TIME}(2^{\log^3n})
		\overset{(2)}{\subseteq} \text{NTIME}(2^{\log^3n})
		$$

		(1) By Problem (b). \\
		(2) By Theorem 1.ii. 

	%%  (d)  %%
	\item NTIME($2^{\log^{3}n}$) ? NTIME($2^{n\log n}$)
		
		To show an inclusion relationship from NTIME $\to$ NTIME, we'd have to
		go through NTIME $\to$ SPACE $\to$ NSPACE $\to$ TIME $\to$ NTIME. For
		NSPACE($f(n)$) $\subseteq$ TIME($2^{g(n)}$) such that $f(n) \in o(g(n))$
		there is no possibility for either inclusion direction, so we cannot say
		based on our knowledge.

	%%  (e)  %%
	\item NTIME($2^{n \log n}$) $\supsetneq$ SPACE($n$)
		\begin{align}
		\text{SPACE}(n) &\subsetneq \text{SPACE}(n\log(\log n))\\% 1
		\text{SPACE}(n\log(\log n))	&\subseteq \text{NSPACE}(n\log(\log n))\\% 2
		\text{NSPACE}(n\log(\log n)) &\subseteq \text{TIME}(2^{n\log n})\\% 3
		\text{TIME}(2^{n\log n}) &\subseteq \text{NTIME}(2^{n\log n})% 4
		\end{align}

		(1) By Theorem 4. Clearly $n \in o(n\log(\log n))$\\
		(2) By Theorem 1.iii. \\
		(3) By Corollary.
		$$
		\lim_{n \to \infty} \frac{n\log(\log n)}{n\log n} 
		= \frac{\log(\log n)}{\log n} = 0
		$$
		(4) By Theorem 1.ii.

	%%  (f)  %%
	\item SPACE($n$) $\supsetneq$ NSPACE($\log^{2} n$)
		$$
		\text{NSPACE}(\log^2n) \overset{(1)}{\subseteq} \text{SPACE}(\log^4n)
		\overset{(2)}{\subsetneq} \text{SPACE}(n)
		$$

		(1) By Theorem 3. \\
		(2) By Theorem 4. Clearly $\log^4n \in o(n)$

	%%  (g)  %%
	\item TIME($2^{\log^{3}n}$) $\subseteq$ NTIME($2^{\log^{3}n}$)
		$$
		\text{TIME}(2^{\log^3n})
		\overset{(1)}{\subseteq} \text{NTIME}(2^{\log^3n})
		$$

		(1) By Theorem 1.ii.

		Cannot be $\subsetneq$, as there would have to be a $g(n)$ greater than
		$2^{\log^3n}$ so that Theorem 5 could apply, but also equal to
		$2^{\log^3n}$ so that Theorem 1.ii can apply. Obviously this is not
		possible. Further, there cannot be an inclusion relationship from NTIME
		$\to$ TIME, as there would have to be an additional exponential factor
		for TIME, from the Corollary.

	%%  (h)  %%
	\item NSPACE($\log^{2}n$) $\subsetneq$ NTIME($2^{n\log n}$)
		\setcounter{equation}{0}
		$$
		\text{NSPACE}(\log^2n) \overset{(1)}{\subseteq} \text{TIME}(2^{\log^3n})
		\overset{(2)}{\subsetneq} \text{TIME}(2^{n\log n})
		\overset{(3)}{\subseteq} \text{NTIME}(2^{n\log n})
		$$
		(1) By Corollary. $\log^2n \in o(\log^3n)$ \\
		(2) By Theorem 5.
		$$
		2^{\log^3n} \in o\left(\frac{2^{n\log n}}{n\log n}\right) \\
		\iff \lim_{n \to \infty} \frac{2^{\log^3n} \cdot n\log n}{2^{n\log n}}=0
		$$
		$2^{n\log n}$ dominates, so the limit holds. \\
		(3) By Theorem 1.ii.

	%%  (i)  %%
	\item NTIME($2^{\log^{3}n}$) ? SPACE($n$)

		From SPACE $\to$ NTIME we would need to go from $n$ to something that
		dominates $n$ in the exponent (going through the Corollary), which
		$\log^3 n$ does not.

		From NTIME $\to$ SPACE by Theorem 1.iv we get SPACE($2^{\log^3n}$), but
		there is no way to get to SPACE($n$) from here.


	%%  (j)  %%
	\item NTIME($2^{n\log n}$) $\supsetneq$ TIME($2^{\log^{3}n}$)
		$$
		\text{TIME}(2^{\log^3n})
		\overset{(1)}{\subsetneq} \text{TIME}(2^{n\log n})
		\overset{(2)}{\subseteq} \text{NTIME}(2^{n\log n})
		$$
		(1) By Theorem 5. (Same proof as (2) in Problem (h)).

		(2) By Theorem 1.ii.
\end{enumerate}
\end{document}
