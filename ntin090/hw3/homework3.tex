\documentclass{article}
\usepackage{amsmath}
\usepackage{tikz}
\usepackage{hyperref}
\usepackage{enumitem}
\usetikzlibrary{chains,fit,shapes}

\author{Andrew McIsaac}
\title{Introduction to Complexity and Computability: Homework 3}
\date{\today}


\begin{document}
\maketitle

\subsection*{Homework Problem 1}

Consider the language $S = \{\langle M_1, M_2\rangle~|~L(M_1) \cap L(M_2) = \emptyset\}$.
Decide whether the language $S$ is a decidable language. If it is not decidable,
decide if $S$ or $\bar{S}$ is a partially decidable language.
\newline
\newline
By Rice's Theorem, language S is undecidable because the language is nontrivial.

Fix $M_2$ so that $L(M_2) = \{01\}$. Then it is nontrivial because for $M_1$ some Turing Machines
accept languages that have an empty intersection with $\{01\}$ and others do not.

It is also clear that the property is a property of the language and not of the 
Turing Machines. If two TMs recognise the same language ($L(M_1') = L(M_1'')$),
then either both have descriptions in $S$ or neither do.
%It is nontrivial because some Turing Machines accept languages that have an empty intersection
%with the language accepted by another TM, but other TMs accept a language that
%share part of the language with the language accepted by another TM.
%
%For example, let $L(M_1) = \{x\}$, and let $L(M_2) = \{y\}$, where $x \neq y$.
%Then $L(M_1) \cap L(M_2) = \emptyset$, and $\langle M_1, M_2\rangle \in S$.
%Now let $L(M_2) = \{x, y\}$, then $L(M_1) \cap L(M_2) \neq \emptyset$, and
%$\langle M_1, M_2\rangle \notin S$.
\newline
\newline
$\bar{S}$ is partially decidable. $\bar{S} = \{\langle M_1, M_2\rangle~|~
L(M_1) \cap L(M_2) \neq \emptyset\}$.

For each $x$ in $\Sigma^*$ in lexicographic order, run $M_1$ on $x$ for one move,
and run $M_2$ on $x$ for one move.
If $M_1$ halts and accepts, add $x$ to the set of strings of the language accepted by $M_1$.
If $M_2$ halts and accepts, add $x$ to the set of strings of the language accepted by $M_2$.

At each step of the computation, the Turing Machines run one more step of all
previous strings, plus the first step of the next $x$. I.e. First, perform the
first move of the TM for the first $x$, then perform the second step of the TM
for the first $x$ and the first step of the TM for the second $x$. Then perform
the third step of the TM for the first $x$, the second step of the TM for the
second $x$, and the first step of the TM for the third $x$, and so on.
If there is a string that is accepted by both $M_1$ and $M_2$ (that is, string $w$
has been added to both the set of strings of the language accepted by $M_1$ and
the set of strings of the language accepted by $M_2$), then $\langle M_1, M_2 \rangle \in \bar{S}$.
\newline
\newline
$S$ is not partially decidable. By Post's Theorem, if both $S$ and $\bar{S}$ are
partially decidable, then $S$ is decidable. But $S$ is undecidable, and $\bar{S}$
is partially decidable, so $S$ must not be partially decidable.

\subsection*{Homework Problem 2}
Consider the language 
$$S = \{\langle M \rangle~|~(\forall w \in \Sigma^*)(w \in L(M) \iff w^R \in L(M))\}.$$

Show that the following holds:
\newline
\newline
(a) $L_u \leq_m S$

Find $$f(\langle M, x \rangle) = \langle M' \rangle$$

Given $M$ and $x$, define $M'$:

On input $w$:
\begin{itemize}
	\itemsep 0em
	\item If $w = 10$, \textit{accept}.
	\item If $w = 01$, run $M$ on $x$. If $M$ accepts, \textit{accept}. 
	\item Otherwise, \textit{reject}.
\end{itemize}

$$
L(M') = \begin{cases}
	\{01, 10\} & \text{if}\ x \in L(M), \\
	\{10\} & \text{otherwise}
\end{cases}
$$

If $M$ accepts x, $L(M') = \{01, 10\}$, so $\langle M' \rangle \in S$. If $M$
does not accept $x, L(M') = \{10\}$, so there is a string $w \in L(M')$ for which
$w^R \notin L(M')$, and $\langle M' \rangle \notin S$. So we have a total
computable function $f(\langle M, x \rangle) = \langle M' \rangle$, and $L_u \leq_m S$.
%$\langle M, x \rangle \in L_u \iff x \in L(M) \iff (\forall w \in \Sigma^*)
%(w \in L(M') \iff w^R \in L(M')) \iff \langle M' \rangle \in S$
\newline
\newline
(b) $L_u \leq_m \bar{S}$

%From the definition of mapping reducibility, it follows that a function $f$ for
%which $L_u \leq_m \bar{S}$ is the same for $\bar{L_u} \leq_m S$, so it is enough
%to show that
%$\bar{L_u} \leq_m S$, where $\bar{L_u} = \{\langle M, x \rangle~|~x \notin L(M)\}.
%
%The proof is similar to part (a).
%
%$$f(\langle M, x \rangle) = \langle M' \rangle$$
%
%Given $M$ and $x$, define $M'$:
%
%On input $w$:
%\begin{itemize}
%	\itemsep 0em
%	\item If $w = 10$, \textit{accept}.
%	\item If $w = 01$, run $M$ on $x$. If $M$ accepts, \textit{reject}. 
%		If $M$ rejects, \textit{accept}.
%	\item Otherwise, \textit{reject}.
%\end{itemize}
%
%$$
%L(M') = \begin{cases}
%	\{01, 10\} & \text{if}\ x \notin L(M), \\
%	\{10\} & \text{otherwise}
%\end{cases}

%Alternative View:

$$\bar{S} = \{\langle M \rangle | (\exists w \in \Sigma^*) (w \in L(M) \iff w^R \notin L(M)\}$$

Given $M$ and $x$, define $M'$:

On input $w$:
\begin{itemize}
	\item If $w = 01$, run $M$ on $x$. If $M$ accepts, \textit{accept}.
	\item Otherwise, \textit{reject}.
\end{itemize}

If $M$ accepts $x$, $01$ is in the language accepted by $M'$ and its reverse
($10$) isn't. So $\langle M' \rangle \in \bar{S}$. If $M$ does not accept $x$,
the language is empty, so there is no string $w$ for which the string is in the
language and its reverse isn't. So $\langle M' \rangle \notin \bar{S}$, and we
have a total computable function $f(\langle M, x \rangle) = \langle M' \rangle$.
So $L_u \leq_m \bar{S}$.


%Given $M$ and $x$, define $M'$:
%
%On input $w$:
%\begin{itemize}
%	\itemsep 0em
%	\item If $w = 10$, accept.
%	\item If $w = 01$:
%\begin{itemize}
%	\itemsep 0em
%	\item Run $M$ on $x$. If $M$ accepts, accept.
%	\item Else, accept.
%\end{itemize}
%	\item Otherwise, reject.
%\end{itemize}
%
%If M on x accepts, $x \in L(M)$, $L(M') = \{10\}$, so $\langle M' \rangle \in \bar{S}$
%
%If M on x rejects, x \notin L(M), 




\end{document}
