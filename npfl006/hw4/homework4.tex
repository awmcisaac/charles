\documentclass[a4paper]{article}

\usepackage[utf8]{inputenc}
\usepackage[T1]{fontenc}
\usepackage{textcomp}
\usepackage{amsmath, amssymb}


% figure support
\usepackage{import}
\usepackage{xifthen}
\pdfminorversion=7
\usepackage{pdfpages}
\usepackage{transparent}
\newcommand{\incfig}[1]{%
	\def\svgwidth{\columnwidth}
	\import{./figures/}{#1.pdf_tex}
}

\pdfsuppresswarningpagegroup=1

\title{NPFL006 Introduction to Formal Linguistics - Homework 4}
\date{\today}
\author{Andrew McIsaac}

\begin{document}
\maketitle

	\begin{enumerate}
		\item What type of graphs are used to represent the meaning of sentences?

			UCCA uses directed acyclic graphs.

			AMR uses rooted, directed graphs.

		\item What do nodes and edges of the graph represent / what do they
			correspond to?

			In UCCA, nodes are units which are either terminal or are several
			elements which represent a single entity with a semantic or
			cognitive consideration. The internal structure of a unit is
			represented by its outbound edges.
			Edges correspond to categories and represent the descendant unit's
			role in forming the semantics of the parent unit.

			AMR nodes represent entities, events, properties and states. Edges
			are relations that link entities.

		\item What labels are used on nodes?

			No labels are used on nodes for UCCA, except for at the leaves of
			the DAG, where the terminals are words or multi-word chunks. Nodes
			can be identified by the labels on their outgoing edges.

			In AMR, there are also only labels on leaf nodes, which are labelled
			with concepts (e.g. ``(b/boy)'' refers to an instance of the concept
			boy).

		\item What labels are used on edges?

			For UCCA, edges are labelled with category abbreviations. For
			example, an edge might be labelled with ``A'' to indicate a
			participant in a scene, or ``H'' to describe a scene linked to another
			scene (a scene describes some movement or action, or a temporally
			persistent state).

			AMR edges are labelled with their specific relations. These can be
			frame arguments, general semantic relations, relations for
			quantities, relations for date-entities, or relations for lists.

	\end{enumerate}
\end{document}
